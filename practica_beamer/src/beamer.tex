\documentclass{beamer}
\usepackage[spanish]{babel}
\usepackage[utf8]{inputenc}
\usepackage{graphicx}

%%%%%%%%%%%%%%%%%%%%%%%%%%%%%

\usetheme{Madrid}
\usetheme{Antibes}
%\usetheme{boxes}
\usetheme{tree}
%\usetheme{classic}
\usecolortheme[RGB={122,59,122}]{structure}


%%%%%%%%%%%%%%%%%%%%%%%%%%%%

\title[Presentación con Beamer]{Clase de Problemas }
\author[Sergio Vega García]{Sergio Vega García}
\date[15-03-2013]{15 de marzo de 2013}

%%%%%%%%%%%%%%%%%%%%%%%%%%%%

\begin{document}
%%%%%%%%%%%%%%

    \begin{frame}
        \begin{scriptsize}
            \begin{center}
                Universidad de La Laguna\cite{ULL}\\
                Técnicas experimentales\cite{TE} \\
                Sergio Vega García
             \end{center}
         \end{scriptsize}
    \end{frame}

%%%%%%%%%%%%%
    
    \begin{frame}
       \frametitle{Índice}  
       \tableofcontents[pausesections]
    \end{frame}

%%%%%%%%%%%%%    
    
\section{Igualdades Trigonométricas}

 \begin{frame}

 \frametitle{Igualdades Trigonométricas}

\begin{center}
  
   $sen^2(x) + cos^2(x) = 1$\\   
   \vspace{1.5cm}
   $tan(x) = \frac{sen(x)}{cos(x)}$
   
  
\end{center}

 

\end{frame}

%%%%%%%%%%%%%%%%

\section{Otras Igualdades}

\begin{frame}
 
 \frametitle{Otras Igualdades}
 
\begin{center}
 
 $f'(x)= lim_{x\to\infty}(\frac{f(x+h)-f(x)}{h})$\\
 \vspace{0.75cm}
 $f(x)= \int f'(x)$\\
 \vspace{0.75cm}
 $\sum_{i=1}^{n}(aX+b) = a\frac{X_1 + X_2 + ... + X_n}{n} + b$
 
\end{center} 
 
\end{frame}

%%%%%%%%%%%%%%%%

\section{Bibliografía}

\begin{frame}
  \frametitle{Bibliografía}

  \begin{thebibliography}{10}

    \beamertemplatebookbibitems
    \bibitem{ULL}
    www.ull.es
    
    \beamertemplatebookbibitems
    \bibitem{TE}
    Materia del Grado en Matemáticas.

  \end{thebibliography}
\end{frame}

%%%%%%%%%%%%%%%%

\end{document}
